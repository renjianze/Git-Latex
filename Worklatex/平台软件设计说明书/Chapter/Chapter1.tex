\chapter{准备阶段(Preamble)}

\section{Intend}
The practice describes the pattern for specifying requirements.

这个文档描述了具有需求。

\subsection{Motivation/Pros}
\begin{enumerate}
	\item Reduction of language effects (misunderstandings);
	\item Supports unambiguousness of syntax;
	\item Supports creation of high quality requirements;
	\item Support time-efficient creation of requirements.
\end{enumerate}
中文解释
\begin{enumerate}[label=\textbullet]
	\item  减少语言影响(误解);
	\item  支持语法的明确性;
	\item  支持高质量需求的创建;
	\item  支持高效的需求创建。
\end{enumerate}

\subsection{Reference(主要参考文档)}

	\begin{enumerate}[label=\alph*.]
		\item   SAEJ1772.pdf\cite{SAE}
		\item   Practice\_Requirement\_Pattern.pdf\cite{Pattern}
		\item Review\_protocol\_Template.xlsm\cite{Review}
	\end{enumerate}
 
\newpage
\section{Elements of a Requirement}
	本小节描述了软件需求规范;

	\begin{table}[h]
		\renewcommand{\arraystretch}{1.3}
		\centering
		\caption{Elements of a Requirement}
		\begin{tabular}{|p{2.5cm}|p{2cm}|p{6cm}|}  
			\specialrule{0.2em}{0pt}{0pt} 
			\centering \textbf{Element}  & \multicolumn{2}{l|}{\textbf{Description}}  \\    
			\hline
			\textcolor{winered}{<event/condition>}  & \multicolumn{2}{p{8cm}|}{The event that shall trigger the \textcolor{mygreen1}{<action>} when it occurs. \newline OR: The condition that shall be fulfilled to conduct the \textcolor{mygreen1}{<action>}} \\
			\hline
			\textcolor{mybule}{<actor>}  & \multicolumn{2}{p{8cm}|}{The acting part, i.e. the one who is obliged to evaluate the \textcolor{winered}{<event/condition>} and conduct \textcolor{mygreen1}{<action>}} \\
			\hline
			\textcolor{mypurple}{<legal binding>}   &  \multicolumn{2}{p{8cm}|}{Key word signifying the relevance of the requirement. Following key words are applicable (in accordance to RFC 2119)} \\
			\cline{2-3}
						&shall       & This word mean that the item is an absolute requirement. \\
						&shall not   & This phrase mean that the item is an absolute prohibition.\\
						&should      &This word mean that there may exist valid reasons to ignore
						the item, but the full implications must be understood and
						carefully weighed before choosing a different course.\\
						&should not           &This phrase mean that there may exist valid reasons
						when the item is acceptable or even useful, but the full
						implications should be understood and the case carefully
						weighed before implementing any behavior described with
						this label.\\
						&may          &This word mean that the item is truly optional.\\
						&will, is           &These keywords identify a statement of fact, not a
						requirement.\\
			\cline{2-3}
						&    \multicolumn{2}{p{8cm}|}{For objects of type “Req-XXX” only keywords “shall” and “shall not” shall
						be used.\newline
						For objects of type other than “Req-XXX”, like “Heading” or
						“Information” only keywords “should”, “should not”, “may”, “will”, and
						“is” shall be used.}\\
			\hline 
			\textcolor{mygreen1}{<action>}& \multicolumn{2}{p{8cm}|}{The actions being conducted by \textcolor{mybule}{<actor>} when the \textcolor{winered}{<event>} occurs or the
			\textcolor{winered}{<condition>} is fulfilled.}\\
			\hline
			\textcolor{mygreen2}{<object of action>} & \multicolumn{2}{p{8cm}|}{Optional specification of the object that undergoes the \textcolor{mygreen1}{<action>}.}\\
			\specialrule{0.2em}{0 pt}{0pt} 
		\end{tabular}
	\end{table}


\subsection{Requirements Pattern}

\subsubsection*{Case1}
	\begin{mybox}
		\centering
		\textcolor{mybule}{<actor>}  \textcolor{mypurple}{<legal binding>} \textcolor{mygreen1}{<action>} \textcolor{mygreen2}{<object of action>}\textcolor{winered}{<event/condition>}
	\end{mybox}
	\begin{enumerate}
		\item \textcolor{mybule}{The system} \textcolor{mypurple}{shall} \textcolor{mygreen1}{turn on} \textcolor{winered}{when the power button is pressed while the system is off}.
		\item \textcolor{mybule}{The system} \textcolor{mypurple}{shall} \textcolor{mygreen1}{switch off} \textcolor{mygreen2}{the lights}, \textcolor{winered}{if the battery signal value is below 20\%}.
		\item \textcolor{mybule}{The system}\textcolor{mypurple}{shall} \textcolor{mygreen1}{stay off} \textcolor{winered}{ while the battery signal value is below 20\% }.
	\end{enumerate}


\subsubsection*{Case2}
	\begin{mybox}
		\centering
		\textcolor{winered}{<event/condition>} \textcolor{mybule}{<actor>} \textcolor{mypurple}{<legal binding>} \textcolor{mygreen1}{<action>} \textcolor{mygreen2}{<object of action>}
	\end{mybox}


	\begin{enumerate}
		\item \textcolor{winered}{When the power button is pressed while the system is off}, \textcolor{mybule}{The system} \textcolor{mypurple}{shall} \textcolor{mygreen1}{turn on}.
		\item \textcolor{winered}{If the battery signal value is below 20\%}, then \textcolor{mybule}{The system} \textcolor{mypurple}{shall}  \textcolor{mygreen1}{switch off} \textcolor{mygreen2}{the lights}.
		\item \textcolor{winered}{While the battery signal value is below 20\%}, \textcolor{mybule}{The system} \textcolor{mypurple}{shall} \textcolor{mygreen1}{stay off}.
	\end{enumerate}





\section{控制导引 (Control Pilot)}
	控制导引电路是确保在将电动汽车(EV)或插电式混合动力汽车(PHEV)连接到电动汽车供电设备(EVSE)时正确操作的主要控制手段。它包括安全有效充电所需的一系列事件和功能。

\subsection{控制导引电路参数和车辆状态}
	控制导引电路参数和车辆状态对于EV和EVSE之间的正确通信和控制至关重要。这些参数确保车辆状态(例如,已连接、准备充电)的正确识别和充电行为的正确执行。

\subsection{控制导引信号}
	控制导引信号用于通信EVSE和车辆的操作状态。具体的占空比由EV/PHEV解释为不同的操作命令:
	\begin{enumerate}
		\item 3-7\%的占空比:有效的数字通信命令。
		\item 8-10\%的占空比:解释为有效的10\%占空比。
		\item ≤85\%的占空比:根据电流(安培数) = 占空比(\%) * 0.6。
		\item 85\%的占空比:根据电流(安培数) = (占空比 - 64) * 2.5。
		\item 97\%的占空比:建议解释为有效的96\%占空比。
	\end{enumerate}

\subsection{控制导引容差}
控制导引信号的总体容差为±2\%,其中EVSE的容差为±0.5\%,EV/PHEV的容差可达±1.5\%。


\section{接近检测 (Proximity Detection)}

\subsection{接近检测电路} 
接近检测电路用于检测连接器插入车辆插口的情况,以防止在车辆移动时对连接器造成损坏。该检测涉及电阻器和连接到连接器锁扣释放执行器的机械开关。此检测可用于满足连接器断开电流限制和带有连接器耦合器的车辆移动的要求。

\subsection{接近检测电路参数 }
接近检测电路包括诸如电阻和开关等组件,确保检测到连接器的存在并提供必要的逻辑以确保安全和操作目的。对于交流(AC)充电,该电路的监测是可选的,但对于直流(DC)充电,该监测是强制性的。

接近检测电路参数具体要求如下:

+5V直流(调节):5.0V(标称值),最大值为5.25V,最小值为4.75V。
各种电阻(R4、R5、R6、R7)的等效负载电阻值,具有指定的标称值、最大值和最小值。
总结
控制导引和接近检测电路是确保EV和EVSE安全有效操作的重要部分。控制导引电路通过定义占空比来管理通信和充电状态,而接近检测电路通过检测连接器的存在来防止潜在的损坏和确保安全。两者都具有特定的参数和容差要求,以确保系统的可靠性和功能性。








