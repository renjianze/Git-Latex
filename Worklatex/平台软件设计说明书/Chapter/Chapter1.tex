\chapter{准备阶段(Preamble)}

\section{Intend}
The practice describes the pattern for specifying requirements.

这个文档描述了具有需求。

\subsection{Motivation/Pros}
\begin{enumerate}
	\item Reduction of language effects (misunderstandings);
	\item Supports unambiguousness of syntax;
	\item Supports creation of high quality requirements;
	\item Support time-efficient creation of requirements.
\end{enumerate}
中文解释
\begin{enumerate}[label=\textbullet]
	\item  减少语言影响(误解);
	\item  支持语法的明确性;
	\item  支持高质量需求的创建;
	\item  支持高效的需求创建。
\end{enumerate}

\subsection{Reference(主要参考文档)}

	\begin{enumerate}
		\item   GB-T18487.1-2015 电动车辆传导充电系统:通用要求\cite{GB18487_1} \textcolor{myred}{(作废)}
		\item 	GB-T18487.2-2017 电动车辆传导充电系统: 非车载传导供电设备电磁兼容要求\cite{GB18487_2}
		\item   GB-T18487.3-2001 电动车辆传导充电系统:电动车辆交流直流充电机(站)\cite{GB18487_3} 
		\item 	GB-T18487.4-2021 电动汽车传导充放电系统:车辆对外放电要求\cite{GB18487_4}  \textcolor{myred}{(未发布)}
		\item   GB-T20234.1-2023 电动汽车充电用连接装置:通用要求\cite{GB20234_1}
		\item   GB-T20234.2-2015 电动汽车充电用连接装置:交流充电接口\cite{GB20234_2}
		\item   GB-T20234.3-2015 电动汽车充电用连接装置:直流充电接口\cite{GB20234_3}
		\item   SAEJ1772.pdf\cite{SAE} $\bigstar \bigstar \bigstar$ 
		\item   IEC61851-1-2010-控制导引电路相关内容\cite{IEC61851} $\bigstar \bigstar \bigstar$
		\item   C001-PHEV项目BOBCDCD变换器总成软件规格书\cite{PHEV}
		\item   Practice\_Requirement\_Pattern.pdf\cite{Pattern}
		\item 	Review\_protocol\_Template.xlsm\cite{Review}
		\item   国内外电动汽车充电系统标准综述\cite{zs}
		\item   10kW电动汽车车载充电机及其软件策略研究\_赵春洋.pdf\cite{zhao}
	\end{enumerate}
 
\newpage
\section{Elements of a Requirement}
	本小节描述了软件需求规范;

	\begin{table}[h]
		\renewcommand{\arraystretch}{1.3}
		\centering
		\caption{Elements of a Requirement}
		\begin{tabular}{|p{2.5cm}|p{2cm}|p{6cm}|}  
			\specialrule{0.2em}{0pt}{0pt} 
			\centering \cellcolor{mygray} \textbf{Element}  & \multicolumn{2}{l|}{\cellcolor{mygray} \textbf{Description}}  \\    
			\hline
			\textcolor{winered}{<event/condition>}  & \multicolumn{2}{p{8cm}|}{The event that shall trigger the \textcolor{mygreen1}{<action>} when it occurs. \newline OR: The condition that shall be fulfilled to conduct the \textcolor{mygreen1}{<action>}} \\
			\hline
			\textcolor{mybule}{<actor>}  & \multicolumn{2}{p{8cm}|}{The acting part, i.e. the one who is obliged to evaluate the \textcolor{winered}{<event/condition>} and conduct \textcolor{mygreen1}{<action>}} \\
			\hline
			\textcolor{mypurple}{<legal binding>}   &  \multicolumn{2}{p{8cm}|}{Key word signifying the relevance of the requirement. Following key words are applicable (in accordance to RFC 2119)} \\
			\cline{2-3}
						&shall       & This word mean that the item is an absolute requirement. \\
						&shall not   & This phrase mean that the item is an absolute prohibition.\\
						&should      &This word mean that there may exist valid reasons to ignore
						the item, but the full implications must be understood and
						carefully weighed before choosing a different course.\\
						&should not           &This phrase mean that there may exist valid reasons
						when the item is acceptable or even useful, but the full
						implications should be understood and the case carefully
						weighed before implementing any behavior described with
						this label.\\
						&may          &This word mean that the item is truly optional.\\
						&will, is           &These keywords identify a statement of fact, not a
						requirement.\\
			\cline{2-3}
						&    \multicolumn{2}{p{8cm}|}{For objects of type “Req-XXX” only keywords “shall” and “shall not” shall
						be used.\newline
						For objects of type other than “Req-XXX”, like “Heading” or
						“Information” only keywords “should”, “should not”, “may”, “will”, and
						“is” shall be used.}\\
			\hline 
			\textcolor{mygreen1}{<action>}& \multicolumn{2}{p{8cm}|}{The actions being conducted by \textcolor{mybule}{<actor>} when the \textcolor{winered}{<event>} occurs or the
			\textcolor{winered}{<condition>} is fulfilled.}\\
			\hline
			\textcolor{mygreen2}{<object of action>} & \multicolumn{2}{p{8cm}|}{Optional specification of the object that undergoes the \textcolor{mygreen1}{<action>}.}\\
			\specialrule{0.2em}{0 pt}{0pt} 
		\end{tabular}
	\end{table}


\subsection{Requirements Pattern}

\subsubsection*{Case1}
	\begin{mybox}
		\centering
		\textcolor{mybule}{<actor>}  \textcolor{mypurple}{<legal binding>} \textcolor{mygreen1}{<action>} \textcolor{mygreen2}{<object of action>}\textcolor{winered}{<event/condition>}
	\end{mybox}
	\begin{enumerate}
		\item \textcolor{mybule}{The system} \textcolor{mypurple}{shall} \textcolor{mygreen1}{turn on} \textcolor{winered}{when the power button is pressed while the system is off}.
		\item \textcolor{mybule}{The system} \textcolor{mypurple}{shall} \textcolor{mygreen1}{switch off} \textcolor{mygreen2}{the lights}, \textcolor{winered}{if the battery signal value is below 20\%}.
		\item \textcolor{mybule}{The system} \textcolor{mypurple}{shall} \textcolor{mygreen1}{stay off} \textcolor{winered}{ while the battery signal value is below 20\% }.
	\end{enumerate}


\subsubsection*{Case2}
	\begin{mybox}
		\centering
		\textcolor{winered}{<event/condition>} \textcolor{mybule}{<actor>} \textcolor{mypurple}{<legal binding>} \textcolor{mygreen1}{<action>} \textcolor{mygreen2}{<object of action>}
	\end{mybox}


	\begin{enumerate}
		\item \textcolor{winered}{When the power button is pressed while the system is off}, \textcolor{mybule}{The system} \textcolor{mypurple}{shall} \textcolor{mygreen1}{turn on}.
		\item \textcolor{winered}{If the battery signal value is below 20\%}, then \textcolor{mybule}{The system} \textcolor{mypurple}{shall}  \textcolor{mygreen1}{switch off} \textcolor{mygreen2}{the lights}.
		\item \textcolor{winered}{While the battery signal value is below 20\%}, \textcolor{mybule}{The system} \textcolor{mypurple}{shall} \textcolor{mygreen1}{stay off}.
	\end{enumerate}


\section{电动汽车的充电模式和连接方式}
	参考国家标准文件: GB-T18487.1-2015.pdf\cite{GB18487_1};
	\subsection{充电模式}
		根据国家标准列举了四种充电模式
		\subsubsection*{模式1 Mode1}
			将电动汽车连接到交流电网(电源)时,在电源侧使用了符合GB 2099.1 和 GB 1002要求的插头插座,在电源侧使用了相线、中性线和接地保护的导体。
		\subsubsection*{模式2 Mode2}
			将电动汽车连接到交流电网(电源)时,在电源侧使用了符合GB 2099.1 和 GB 1002要求的插头插座,在电源侧使用了相线、中性线和接地保护的导体,
			并且在充电连接时使用了缆上控制与保护装置(IC-CPD)。
		\subsubsection*{模式3 Mode3}
			将电动汽车连接到交流电网(电源)时,使用了专用供电设备,将电动汽车与交流电网直接连接,并且在专用供电设备上安装了控制引导装置。 \textcolor{red}{$\bigstar  \bigstar \bigstar $}
		\subsubsection*{模式4 Mode4}
			将电动汽车连接到交流电网或直流电网时,使用了带控制导引功能的直流供电设备。 

	\subsection{连接方式}
		连接方式:使用电缆和连接器将电动汽车接入电网(电源)的方法。
		\subsubsection*{连接方式 1}
			将电动汽车和交流电网连接时,使用和电动汽车永久连接在一起的充电电缆和供电插头。
		\subsubsection*{连接方式 2}
			将电动汽车和交流电网连接时,使用带有车辆插头和供电插头的独立的活动电缆组件。
		\subsubsection*{连接方式 3}
			将电动汽车和交流电网连接时,使用了和供电设备永久连接在一起的充电电缆和车辆插头。








