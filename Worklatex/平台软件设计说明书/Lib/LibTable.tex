

\chapter{表格}


\section{三线表}
介绍了最常用的三线表,定义了最基础的三线表,\textcolor{winered}{\textbackslash toprule}、\textcolor{winered}{\textbackslash midrule}和\textcolor{winered}{\textbackslash bottomrule}分别
画顶线中线和底线。

\begin{lstlisting}
     \begin{table}[!htbp]
          \renewcommand{\arraystretch}{1.3}
          \centering
          \caption{三线表}
          \begin{tabular}{ccc}   
               \toprule
               \centering{单词}  & 缩写  & 中文\\    
               \midrule
               Electric Vehicle  & EV  & 电动汽车\\
               Plug in Hybrid Electric Vehicles  & PHEV & 插电混动\\
               Society of Automotive Engineers   & SAE  & 国际汽车工程协会\\
               \bottomrule
          \end{tabular}
     \end{table}
\end{lstlisting}

\begin{table}[!htbp]
     \renewcommand{\arraystretch}{1.3}
     \centering
     \caption{名称定义}
     \begin{tabular}{ccc}   
          \toprule
          \centering{单词}  & 缩写  & 中文\\    
          \midrule
          Electric Vehicle  & EV  & 电动汽车\\
          Plug in Hybrid Electric Vehicles  & PHEV & 插电混动\\
          Society of Automotive Engineers   & SAE  & 国际汽车工程协会\\
          \bottomrule
     \end{tabular}
\end{table}





\section{通用表格}
三线表格式,控制每个列的宽度。
     \begin{enumerate}[leftmargin=2.5cm]
          \item p\{<width>\}  :左对齐;
          \item >\{\textbackslash centering\}p\{<width>\}: 居中;
          \item >\{\textbackslash raggedleft \textbackslash arraybackslash\}p\{<width>\}:右对齐;
          \item \textbackslash specialrule{0.2em}{0pt}{0pt}:设置线宽;
          \item \textbackslash renewcommand{\arraystretch}{1.3}:设置行高。
     \end{enumerate}

举例:

    \begin{lstlisting}
     \begin{table}[!htbp]
          \renewcommand{\arraystretch}{1.3}
          \centering
          \caption{通用表格}
          \begin{tabular}{|p{6cm}|>{\centering\arraybackslash}p{2cm}|>{\centering\arraybackslash}p{3cm}|}  
              \centering{单词}  & 缩写  & 中文\\    
              \hline
              Electric Vehicle  & EV  & 电动汽车\\
              Plug in Hybrid Electric Vehicles  & PHEV & 插电混动\\
              Society of Automotive Engineers   & SAE  & 国际汽车工程协会\\
              \specialrule{0.2em}{0pt}{0pt} 
          \end{tabular}
      \end{table}
    \end{lstlisting}

    \begin{table}[!htbp]
          \renewcommand{\arraystretch}{1.3}
          \centering
          \caption{通用表格}
          \begin{tabular}{|p{6cm}|>{\centering\arraybackslash}p{2cm}|>{\centering\arraybackslash}p{3cm}|}  
               \specialrule{0.2em}{0pt}{0pt} 
               \centering{单词}  & 缩写  & 中文\\    
               \hline
               Electric Vehicle  & EV  & 电动汽车\\
               Plug in Hybrid Electric Vehicles  & PHEV & 插电混动\\
               Society of Automotive Engineers   & SAE  & 国际汽车工程协会\\
               \specialrule{0.2em}{0pt}{0pt} 
          \end{tabular}
     \end{table}


\section{表中表}


\begin{lstlisting}
     \centering
     \caption{这是我的表格}
     \renewcommand{\arraystretch}{1.5}
     \begin{tabular}{|c|c|c|c|}
          \specialrule{0.1em}{0pt}{0pt}
          \multirow{2}{*}{内容} & \multicolumn{3}{c|}{标题} \\  %跨行表和跨列表格
          \cline{2-4}
          &苹果 & 相机  & 西瓜\\
          \hline
          重量  & 2 $\mathrm{kg}$ & 3 $\mathrm{kg}$ & 4  $\mathrm{kg}$\\
          体积 & 3 &5&8\\
          面积 & 5 &5 &5\\
          \hline  
     \end{tabular}
\end{lstlisting}


\begin{table}[!htbp]
     \centering
     \caption{这是我的表格}
     \renewcommand{\arraystretch}{1.5}
     \begin{tabular}{|c|c|c|c|}
          \specialrule{0.1em}{0pt}{0pt}
          \multirow{2}{*}{内容} & \multicolumn{3}{c|}{标题} \\
          \cline{2-4}
          &苹果 & 相机  & 西瓜\\
          \hline
          重量  & 2 $\mathrm{kg}$ & 3 $\mathrm{kg}$ & 4  $\mathrm{kg}$\\
          体积 & 3 &5&8\\
          面积 & 5 &5 &5\\
          \hline  
     \end{tabular}
\end{table}

\section{复杂表格合并}
\begin{table}[!htbp]
     \renewcommand{\arraystretch}{1.3}
     \centering
     \caption{Elements of a Requirement}
     \begin{tabular}{|p{2.5cm}|p{2cm}|p{6cm}|}  
         \specialrule{0.2em}{0pt}{0pt} 
         \centering \textbf{Element}  & \multicolumn{2}{l|}{\textbf{Description}}  \\    
         \hline
         \textcolor{winered}{<event/condition>}  & \multicolumn{2}{p{8cm}|}{The event that shall trigger the \textcolor{mygreen1}{<action>} when it occurs. \newline OR: The condition that shall be fulfilled to conduct the \textcolor{mygreen1}{<action>}} \\
         \hline
         \textcolor{mybule}{<actor>}  & \multicolumn{2}{p{8cm}|}{The acting part, i.e. the one who is obliged to evaluate the \textcolor{winered}{<event/condition>} and conduct \textcolor{mygreen1}{<action>}} \\
         \hline
         \textcolor{mypurple}{<legal binding>}   &  \multicolumn{2}{p{8cm}|}{Key word signifying the relevance of the requirement. Following key words are applicable (in accordance to RFC 2119)} \\
            \cline{2-3}
                       &shall       & This word mean that the item is an absolute requirement. \\
                          &shall not   & This phrase mean that the item is an absolute prohibition.\\
                          &should      &This word mean that there may exist valid reasons to ignore
                          the item, but the full implications must be understood and
                          carefully weighed before choosing a different course.\\
                          &should not           &This phrase mean that there may exist valid reasons
                          when the item is acceptable or even useful, but the full
                          implications should be understood and the case carefully
                          weighed before implementing any behavior described with
                          this label.\\
                          &may          &This word mean that the item is truly optional.\\
                          &will, is           &These keywords identify a statement of fact, not a
                          requirement.\\
           \cline{2-3}
                          &    \multicolumn{2}{p{8cm}|}{For objects of type “Req-XXX” only keywords “shall” and “shall not” shall
                          be used.\newline
                          For objects of type other than “Req-XXX”, like “Heading” or
                          “Information” only keywords “should”, “should not”, “may”, “will”, and
                          “is” shall be used.}\\
           \hline 
           \textcolor{mygreen1}{<action>}& \multicolumn{2}{p{8cm}|}{The actions being conducted by \textcolor{mybule}{<actor>} when the \textcolor{winered}{<event>} occurs or the
           \textcolor{winered}{<condition>} is fulfilled.}\\
           \hline
           \textcolor{mygreen2}{<object of action>} & \multicolumn{2}{p{8cm}|}{Optional specification of the object that undergoes the \textcolor{mygreen1}{<action>}.}\\
         \specialrule{0.2em}{0 pt}{0pt} 
     \end{tabular}
 \end{table}








      
