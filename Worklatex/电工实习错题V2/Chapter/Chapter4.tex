
\chapter{考试易错题}
	\section{第一次考试}
		\subsection{判断题——低压}
			\begin{enumerate}
				\kaishu
				\item \Tquestion{电缆保护层的作用是保护电缆。}
				\item \Tquestion{低压绝缘耐热材料的耐压等级一般为500 V。}
				\item \Tquestion{铁壳开关可用于不频繁启动28kW以下的三相异步电动机。}
				\item \Fquestion{目前我国生产的接触器额定电流一般大于或等于630A。}
				\item \Tquestion{剩余电流动作保护装置主要用于1000V以下的低压系统。}
				\item \Fquestion{时间继电器的文字符号为KM。} \\ \jiexi{把题库中的KT改成了KM。题库没有原题。}
				\item \Tquestion{交流接触器的额定电流,是在额定的工作条件下所决定的电流值。}
				\item \Fquestion{电动机的短路试验是给电机施加35V左右的电压。} \\ \jiexi{题库没有这个题。}
				\item \Fquestion{刀开关在作隔离开关选用时,要求刀开关的额定电流要大于或等于线路实际的故障电流。} \\ \jiexi{不是实际故障电流}
				\item \Fquestion{}
				\item \Fquestion{}
			\end{enumerate}


		\subsection{选择题——低压}
			\begin{enumerate}
				\kaishu
				\item ( )仪表的灵敏度和精确度较高,多用来制作携带式电压表和电流表。
					\begin{tasks}(\choosenum)
						\task	磁电式
						\task	电磁式
						\task   电动式
					\end{tasks} \daan{A}{磁电式易携带万用表;电磁式精度低;电动式精度高。}

				\item 钳形电流表是利用( )的原理制造的。
					\begin{tasks}(\choosenum) 
						\task 电流互感器
						\task 电压互感器
						\task 变压器
					\end{tasks} \daan{A}

				\item 静电引起爆炸和火灾的条件之一是( )。
				 	\begin{tasks}(\choosenum) 
						\task 有爆炸性混合物存在
						\task 静电能量要足够大
						\task 有足够的温度
					\end{tasks} \daan{A}

				\item 钳形电流表由电流互感器和带( )的磁电式表头组成。
					\begin{tasks}(\choosenum) 
						\task 测量电路
						\task 整流装置
						\task 指针
					\end{tasks} \daan{B}
				
				\item 带“回”字符号标志的手持电动工具是( )工具
					\begin{tasks}(\choosenum) 
						\task I类
						\task II类
						\task III类
					\end{tasks} \daan{B}{题库没看到这题。}

				\item 将一根导线均匀拉长原来的两倍,它的电阻为原阻值的( )倍。
					\begin{tasks}(\choosenum) 
						\task 1
						\task 2
						\task 4
				\end{tasks} \daan{C}{$R = \rho \frac{l}{S}$,为什么$S$变小两倍?}

				\item 测量电动机线圈对地的绝缘电阻时,摇表的“L"、 “E"两个接线柱应( )。
					\begin{tasks}(\choosenum) 
						\task "E接在电动机出线的端子, “L“接电动机的外壳
						\task "L"接在电动机出线的端子, “E"接电动机的外壳 
						\task 随便接,没有规定
					\end{tasks} \daan{B}

				% \item 1
				% 	\begin{tasks}(1) 
				% 		\task 0
				% 		\task 0
				% 		\task 0
				% 	\end{tasks} \daan{A}
			\end{enumerate}


			\subsection{判断题—安全}
			\begin{enumerate}
				\kaishu
				\item \Fquestion{应急管理部门和其他负有安全生产监督管理职责的部门可以进入生产经营单位进行检查,调阅所有资料,向有关单元和人员了解情况。}
				\item \Tquestion{低压绝缘耐热材料的耐压等级一般为500 V。}
				\item \Tquestion{铁壳开关可用于不频繁启动28kW以下的三相异步电动机。}
				% \item \Fquestion{目前我国生产的接触器额定电流一般大于或等于630A。}
				% \item \Tquestion{剩余电流动作保护装置主要用于1000V以下的低压系统。}
				% \item \Fquestion{时间继电器的文字符号为KM。} \\ \jiexi{把题库中的KT改成了KM。题库没有原题。}
				% \item \Tquestion{交流接触器的额定电流,是在额定的工作条件下所决定的电流值。}
				% \item \Fquestion{电动机的短路试验是给电机施加35V左右的电压。} \\ \jiexi{题库没有这个题。}
				% \item \Fquestion{刀开关在作隔离开关选用时,要求刀开关的额定电流要大于或等于线路实际的故障电流。} \\ \jiexi{不是实际故障电流}
				% \item \Fquestion{}
				% \item \Fquestion{}
			\end{enumerate}

			
		\subsection{选择题——安全}
		\begin{enumerate}
			\kaishu
			\item 以下不属于消防工作贯彻的方针的方针是( )。
				\begin{tasks}(\choosenum)
					\task	安全第一
					\task	预防为主
					\task   防消结合
				\end{tasks} \daan{A}{}

			\item 根据《安全生产法》,安全生产工作应当坚持安全发展,坚持( )的方针。
				\begin{tasks}(\choosenum)
					\task	安全第一、预防为主、综合治理
					\task	安全生产人人有责
					\task   安全为了生产,生产必须安全
				\end{tasks} \daan{A}{}
		\end{enumerate}
				


































