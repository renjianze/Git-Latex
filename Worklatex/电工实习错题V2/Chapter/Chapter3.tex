

\chapter{选择题(易错题)}

	\section{选择题1}
	\begin{enumerate}
		\kaishu
				\item 每一照明(包括风扇)支路总容量一般不大于( ) kW。
						\begin{tasks}(\choosenum)
							\task 2
							\task 3
							\task 4
						\end{tasks}		 \daan{B}{3 kW}

				\item 一般照明线路中,无电的依据是( )。
						\begin{tasks}(\choosenum)
							\task 用摇表测量
							\task 用电笔验电
							\task 用电流表测量
						\end{tasks}		 \daan{B}{电笔验电}	

				\item 用万用表测量电阻时,黑表笔接{\color{red}表内电源}( )。
						\begin{tasks}(\choosenum)
							\task 负极
							\task 两极
							\task 正极
						\end{tasks}		 \daan{C}{红进黑出 ????}			
				
				\item 在一个闭合回路中,电流强度与电源电动势成正比,与电路中内电阻和外电阻之和成反比,这一定律称为( )。
						\begin{tasks}(\choosenum)
							\task 全电路欧姆定律
							\task 全电路电流定律
							\task 部分电路欧姆定律
						\end{tasks}		 \daan{A}{}	

				\item 万用表由表头、( )及转换开关三个主要部分组成。
				\begin{tasks}(\choosenum)
					\task 测量电路
					\task 线圈
					\task 指针
				\end{tasks}		 \daan{A}{万用表由表头、测量电路和转换开关组成。}	

% \newpage
				\item 使用竹梯时,梯子与地面的夹角以( )$^\circ$为宜。
					\begin{tasks}(\choosenum)
						\task 50
						\task 60
						\task 70
					\end{tasks}		 \daan{B}{60°的梯子}	


				\item 导线接头电阻要足够小,与同长度同截面导线的电阻比不大于( )。
					\begin{tasks}(\choosenum)
						\task 1
						\task 1.5
						\task 2
					\end{tasks}		 \daan{A}{}	

				\item 摇表的两个主要组成部分是手摇( )和磁电式流比计。
					\begin{tasks}(\choosenum)
						\task 电流互感器
						\task 直流发电机
						\task 交流发电机
					\end{tasks}		 \daan{B}{暂无解析}	

				\item 电流从左手到双脚引起心室颤动效应,一般认为通电时间与电流的乘积大于( ) mA$\cdot$S时就有生命危险。
					\begin{tasks}(\choosenum)
						\task 30
						\task 16
						\task 50
					\end{tasks}		 \daan{C}{}	


				\item 拉开闸刀时,如果出现电弧,应该( )。
					\begin{tasks}(\choosenum)
						\task 迅速拉开
						\task 立即合闸
						\task 缓慢拉开
					\end{tasks}		 \daan{A}{开闸时出现电弧应该迅速拉开。}	

				\item 照明系统中的每一单相回来中,灯具与插座的数量不宜超过( )个。
					\begin{tasks}(\choosenum)
						\task 20
						\task 25
						\task 30
					\end{tasks}		 \daan{B}{不超过25个}	

				\item 墙边开关安装时距离地面的高度为( )mA。
					\begin{tasks}(\choosenum)
						\task 1.3
						\task 1.5
						\task 2
					\end{tasks} \daan{A}{}

				\item 热继电器的整定电流为电动机额定电流的( )\%。
					\begin{tasks}(\choosenum)
						\task 100
						\task 120
						\task 130
				\end{tasks} \daan{A}{额定电流的0.95$\sim$1.05倍之间调节。}
				
				\item 为避免高压变配电站遭受直击雷,引发大面积停电事故,一般可用( )来防雷。
					\begin{tasks}(\choosenum)
						\task 接闪杆
						\task 阀型避雷器
						\task 接闪网
					\end{tasks} \daan{A}{}	
				
				\item 碳在自然界中有金刚石和石墨两种存在形式,其中石墨是( )。
					\begin{tasks}(\choosenum)
						\task 绝缘体
						\task 导体
						\task 半导体
					\end{tasks} \daan{B}{石墨是导体}
					
				\item 下列灯具中功率因数最高的是( )。
					\begin{tasks}(\choosenum)
						\task 白炽灯
						\task 节能灯
						\task 日光灯
					\end{tasks} \daan{A}{白炽灯是纯电阻的,电压和电流始终同相位。}
				
				\item 异步电动机在启动瞬间,转子绕组中感应的电流很大,使定子流过的启动电流也很大,约为额定电流的( )倍。
					\begin{tasks}(\choosenum)
						\task 2
						\task 4至7
						\task 9至10
					\end{tasks} \daan{B}{}
				
				\item 国家标准规定凡( )kW以上的电动机均采用三角形接法。
					\begin{tasks}(\choosenum)
						\task 3
						\task 4
						\task 7.5
					\end{tasks} \daan{B}{$\triangle$接法比Y形接法电流和功率大三倍。不大于3 kW的三相异步电动机采用Y,大于4 kW的三相异步电动机采用$\triangle$。 ?}

				\item {\color{red}{\ding{72}}}( )仪表可直接用于交、直流测量,且精确度高。
					\begin{tasks}(\choosenum)
						\task 电磁式
						\task 磁电式
						\task 电动式
					\end{tasks} \daan{C}{电动式可测交流和直流,精度高。}

				\item {\color{red}{\ding{72}}}( )仪表可直接用于交、直流测量,但精度低。
					\begin{tasks}(\choosenum)
						\task 磁电式
						\task 电磁式
						\task 电动式
					\end{tasks} \daan{B}{电磁式可测交流和直流,精度低。}
				
				\item {\color{red}{\ding{72}}}万用表的实质是一个带有整流器的( )仪表。
					\begin{tasks}(\choosenum)
						\task 磁电式
						\task 电磁式
						\task 电动式
					\end{tasks} \daan{A}{}

				\item {\color{red}{\ding{72}}}( )仪表由固定的永久磁铁,可转动的线圈及转轴、游丝、指针、机械调零机构等组成。
					\begin{tasks}(\choosenum)
						\task 磁电式
						\task 电磁式
						\task 电动式
					\end{tasks} \daan{A}{磁电式}

				\item 绝缘材料的耐热等级为E级时,其极限工作温度为( ) $^\circ$C。
					\begin{tasks}(\choosenum)
						\task 105
						\task 90
						\task 120
					\end{tasks} \daan{C}{绝缘材料为E级时,极限工作温度120 $^\circ$C}

				\item {\color{red}{\ding{115}}} 熔断器的额定电流( )电动机的起动电流。
					\begin{tasks}(\choosenum)
						\task 等于
						\task 大于
						\task 小于
					\end{tasks} \daan{C}{??}
					

				\item 穿管导线内最多允许( )个导线接头。
					\begin{tasks}(\choosenum)
						\task 1
						\task 2
						\task 0
					\end{tasks} \daan{C}{穿管导线内不允许有接头。}

				\item 在对380 V电机各绕组的绝缘检查中,发现绝缘电阻( ), 则可初步判定为电动机受潮所致,应对电机进行烘干处理。
					\begin{tasks}(\choosenum)
						\task 大于0.5 M$\Omega$
						\task 小于10  M$\Omega$
						\task 小于0.5 M$\Omega$
					\end{tasks} \daan{C}{}

				\item {\color{red}\ding{72}\ding{72}}检查绝缘电阻,一般的漏电保护器需要在进出线端子间,各接线端子与外壳间,接线端子之间进行绝缘电阻测量,其绝缘电阻值应不低于( )。
					\begin{tasks}(\choosenum)
						\task 0.5 M$\Omega$
						\task 1   M$\Omega$
						\task 1.5 M$\Omega$
						\task 2  M$\Omega$
					\end{tasks} \daan{C}{}

				\item  对电机各绕组的绝缘检查,要求是:电动机每1kV工作电压,绝缘电阻( )。
					\begin{tasks}(\choosenum)
						\task 小于0.5 M$\Omega$
						\task 大于等于1  M$\Omega$
						\task 等于0.5 M$\Omega$
					\end{tasks} \daan{B}{}

				\item 在三相对称交流电源星形连接中,线电压超前于所对应的相电压( )$^\circ$。
					\begin{tasks}(\choosenum)
						\task 120
						\task 30
						\task 60
					\end{tasks} \daan{B}{Y连接:U线 = $\sqrt{3}$ U相,超前 30$^\circ$}
				
				\item 电业安全工作规程上规定,对地电压为( )V及以下的设备为低压设备。
					\begin{tasks}(\choosenum)
						\task 380
						\task 400
						\task 250
					\end{tasks} \daan{C}{}

				\item 频敏变阻器其构造与三相电抗相拟,即由三个铁芯柱和( )绕组组成。
					\begin{tasks}(\choosenum)
						\task 二个
						\task 一个
						\task 三个
					\end{tasks} \daan{C}{频敏变阻器是特殊的三相铁芯电抗器,每个柱上有一个绕组。}

				\item 当一个熔断器保护一只灯时,熔断器应串联在开关( )。
					\begin{tasks}(\choosenum)
						\task 前
						\task 后
						\task 中
					\end{tasks} \daan{B}{熔断器串联在开关后}

				\item 交流10 kV母线电压是指交流三相三线制的( )。
					\begin{tasks}(\choosenum)
						\task 线电压
						\task 相电压
						\task 线路电压
					\end{tasks} \daan{A}{母线电压是线电压}

				\item 电流继电器使用时其吸引线圈直接或通过电流互感器( )在被控电路中。
					\begin{tasks}(\choosenum)
						\task 并联
						\task 串联
						\task 串联或并联
					\end{tasks} \daan{B}{}

				\item 星-三角降压启动,是起动时把定子三相绕组作( )联接。
					\begin{tasks}(\choosenum)
						\task 三角形
						\task 星形
						\task 延边三角形
					\end{tasks} \daan{B}{}

				\item 装设接地线,当检验明确无电压后,应立即将检修设备接地并( )短路。
					\begin{tasks}(\choosenum)
						\task 两相
						\task 单相
						\task 三相
					\end{tasks} \daan{C}{}

				\item 体人内约体测体为内电体电阻为内电电为阻为测阻约内电为内测() $\Omega$。 \\    {\color{purple}{复制乱码:人体内电阻}}
					\begin{tasks}(\choosenum)
						\task 300
						\task 200
						\task 500
					\end{tasks} \daan{C}{人体电阻的大小是影响触电后人体受到伤害程度的重要物理因素。人体电阻由(体内电阻)
					和(皮肤)组成,体内电阻基本稳定,约为500 $\Omega$。接触电压为220 V时,人体电阻的平均值为1900 $\Omega$;接触电压为380 V
					时,人体电阻降为1200 $\Omega$。经过对大量实验数据的分析研究确定,人体电阻的平均值一般为2000 $\Omega$左右,而在
					计算和分析时,通常取下限值1700 $\Omega$。}

				\item “禁险制险止登攀测高的登标,攀高攀制危压压牌危牌险高测险!”标险的志登标攀志登牌登登应登的标制的的制作志为攀牌( )。
					\begin{tasks}(\choosenum)
						\task 红底白字
						\task 白底红字
						\task 白底红字黑边
					\end{tasks} \daan{C}{}

				\item  {\color{red}{\ding{72}}}高压验电器的发光电压不应高于额定电压的( )\%。
					\begin{tasks}(\choosenum)
						\task 25
						\task 70
						\task 75
					\end{tasks} \daan{A}{验电器靠近带电体就能发光}

				\item 熔断断器的保护特性称为( )。
					\begin{tasks}(\choosenum)
						\task 灭弧特性
						\task 安秒特性
						\task 时间性
					\end{tasks} \daan{B}{}

				\item 在电气线路安装时,导线与导线或导线与电气螺栓之间的连接最易引发火灾的连接工艺是( )。
					\begin{tasks}(\choosenum)
						\task 铜线与铝线铰接  
						\task 铝线与铝线铰接 
						\task 铜铝过渡接头压接
					\end{tasks} \daan{A}{}

				\item 钳形电流表使用时应先用较大量程,然后在视被测电流的大小变换量程,切换量程时应( )。
					\begin{tasks}(\choosenum)
						\task 直接转动量程开关 
						\task 先退出导线,再转动量程开关 
						\task 一边进线一边换挡
					\end{tasks} \daan{B}{}

				\item 螺旋式熔断器的进线端( )。
					\begin{tasks}(\choosenum)
						\task 上端
						\task 下端
						\task 前端
					\end{tasks} \daan{B}{螺旋式熔断器由瓷帽、熔断管、瓷套、上接线座和下接线座及瓷底座等部分组成。正确的操作是螺旋式熔断器的上触头和旋盖相连,当施工人员更换烧断的熔芯时会更安全。如果把进线接到了上触头的话,那么,旋盖就总是带电了,如果操作不当会发生事故。}

				\item 据一些资料表明,心脏在停止呼吸( ) min内抢救,救活的概率为80 \%。
					\begin{tasks}(\choosenum)
						\task 1
						\task 2
						\task 3
					\end{tasks} \daan{A}{}

				\item 在对可能存在较高跨步电压的接地故障点进行检查时,室内不得接近故障点( ) m以内。
					\begin{tasks}(\choosenum)
						\task 3
						\task 2
						\task 4
					\end{tasks} \daan{C}{室内不得接近故障点4米以内,室外不得接近故障点8米以内。}

				\item 接地线应用多股软裸铜线,其截面积不得小于( ) mm$^2$。
					\begin{tasks}(\choosenum)
						\task 10
						\task 6
						\task 25
					\end{tasks} \daan{C}{}

				\item 三相笼形异步电动机的启动方式有两类,既在额定电压下的直接启动和( )启动。
					\begin{tasks}(\choosenum)
						\task 转子串频敏
						\task 转子串电阻
						\task 降低启动电压
					\end{tasks} \daan{C}{}

				\item  电气火灾发生时,应先切断电源再扑救,但不知或不清楚开关在何处时,应剪断电线,剪切时要( )。
					\begin{tasks}(\choosenum)
						\task 几根线迅速同时切断
						\task 不同相线在不同位置剪断
						\task 在同一位置一根一根剪断
					\end{tasks} \daan{B}{}

				\item 断路器是通过手动或电动等操作机构使断路器合闸,通过( )装置使断路器自动跳闸,到达故障保护的目的( )。
					\begin{tasks}(\choosenum)
						\task 活动
						\task 自动
						\task 脱落
					\end{tasks} \daan{C}{}

			

				\item 电感式日光灯镇流器的内部是( )。
					\begin{tasks}(\choosenum)
						\task 电子电路
						\task 线圈
						\task 振荡电路
					\end{tasks} \daan{B}{电感镇流器是一个铁芯电感线圈。}

				\item 具有反时限安秒特性的元件就具备短路保护和( )保护能力。
					\begin{tasks}(\choosenum)
						\task 机械
						\task 温度
						\task \ding{45}过载
					\end{tasks} \daan{C}{\ding{45}\ding{41}\ding{43}}

				\item 1 kV以上的电容器组采用( )电压互感器接成三角形作为放电装置。
					\begin{tasks}(\choosenum)
						\task 电炽灯
						\task 电流互感器
						\task 电压互感器
					\end{tasks} \daan{C}{}


				\item 刀开关一般用于空载操作,也可用于控制不经常启动的容量小于3 kW的异步电动机,当用于启动异步电动机时,其额定电流应不小于电动机额定电流的( )。
					\begin{tasks}(\choosenum)
						\task 1倍
						\task 2倍
						\task 3倍
						\task 4倍
					\end{tasks} \daan{C}{}

				\item 漏电保护器在安装前,有条件的最好进行动作特性参数测试,安装投入使用后,在使用过程中应( )检验一次,以保证其始终能可靠运行。
					\begin{tasks}(\choosenum)
						\task 每周
						\task 半月
						\task 每个月
						\task 每季度
					\end{tasks} \daan{C}{}

				
				\item 电力系统电压互感器的二次测额定电压均为( ) V。
					\begin{tasks}(\choosenum)
						\task 220
						\task 380
						\task 36
						\task 100
					\end{tasks} \daan{D}{}


				
				\item 交流接触器的电寿命约为机械寿命的( )倍。
					\begin{tasks}(\choosenum)
						\task 1
						\task 10
						\task 1/20
					\end{tasks} \daan{C}{接触器的机械寿命有100万次电器寿命有10万次。}

				\item 接地电阻测量仪主要由手摇发电机、( )、电位器,以及检流计组成。         %%%%  666666
					\begin{tasks}(\choosenum)
						\task 电流互感器
						\task 电压互感器
						\task 变压器	
					\end{tasks} \daan{A}{\ding{78}接地电阻测量仪主要由手摇发电机、电流互感器、电位器,及检流计组成。}

				\item 对于低压配电网,配电容量在100 kW以下时,设备保护接地的接地电阻不应超过( ) $\Omega$。                  %%%%  666666
					\begin{tasks}(\choosenum)
						\task 10
						\task 6
						\task 4
					\end{tasks}		 \daan{A}{10 $\Omega$}	
				
				\item 避雷针是常用的避雷装置,安装时,避雷针宜设独立的接地装置,如果在非高电阻率地区,其接地电阻不宜超过( ) $\Omega$。        %%%%  666666
					\begin{tasks}(\choosenum)
						\task 2
						\task 4
						\task 10
					\end{tasks} \daan{C}{}

				\item 防静电的接地电阻要求不大于( ) $\Omega$。               %%%%  666666
					\begin{tasks}(\choosenum)
						\task 40
						\task 10
						\task 100
					\end{tasks} \daan{C}{}

				\item {\color{red}\ding{72}\ding{72}}施工现场临时用电系统的重复接地电阻不应大于( )。    %%%%  666666
					\begin{tasks}(\choosenum)
						\task 4 $\Omega$
						\task 10   $\Omega$
						\task 20 $\Omega$
						\task 30  M$\Omega$
					\end{tasks} \daan{B}{}

				\item 漏电保护断路器在设备正常工作时,电路电流的相量和( ),开关保持闭合状态。
					\begin{tasks}(\choosenum)
						\task 为正
						\task 为负
						\task 为零
					\end{tasks} \daan{C}{}


				
				\item 在民用建筑物的配电系统中,一般采用( )断路器。
					\begin{tasks}(\choosenum)
						\task 电动式
						\task 框架式
						\task 漏电保护
					\end{tasks} \daan{C}{}


				\item
				下列属于顺磁材料的是( )。
					\begin{tasks}(\choosenum)
						\task 铜
						\task 水
						\task 空气
					\end{tasks} \daan{C}{}


				
				\item 几种线路同杆架设时,必须保证高压线路在低压线路( )。
					\begin{tasks}(\choosenum)
						\task 右方
						\task 左方
						\task 上方
					\end{tasks} \daan{C}{}


				
				\item 三线四线制的零线的截面积一般( )相线截面积。
					\begin{tasks}(\choosenum)
						\task 大于
						\task 小于
						\task 等于
					\end{tasks} \daan{B}{}


				
				\item TN-S俗称( )。
					\begin{tasks}(\choosenum)
						\task 三相四线
						\task 三相五线
						\task 三相三线
					\end{tasks} \daan{B}{}


				
				\item 相线应接在螺口灯头( )。
					\begin{tasks}(\choosenum)
						\task 中心端子
						\task 螺纹端子
						\task 外壳
					\end{tasks} \daan{A}{}

				\item 在易燃、易爆危险场所,电气线路应采用( )或者铠装电缆敷设。
					\begin{tasks}(\choosenum)
						\task 穿金属蛇皮管再沿铺砂电缆沟
						\task 穿水煤气管
						\task 穿钢管
					\end{tasks} \daan{C}{}

			\item 1千伏以上电容器组采用( )接成三角形作为放电装置。
					\begin{tasks}(\choosenum)
						\task 电炽灯
						\task 电流互感器
						\task 电压互感器
					\end{tasks} \daan{C}{}

						% \item N.A.
				% 	\begin{tasks}(\choosenum)
				% 		\task N.A.
				% 		\task N.A.
				% 		\task N.A.
				% 	\end{tasks} \daan{A}{}






	\end{enumerate}





	\section{选择题2(数字)}
			\begin{enumerate}
				\item 照明线路熔断器的熔体额定电流取线路计算电流的( )倍。
					\begin{tasks}(\choosenum)
						\task 0.9
						\task 1.1
						\task 1.5
					\end{tasks} \daan{B}{}

				\item 测量接地电阻时,电位探针应接在距接地端( )m的地方。
					\begin{tasks}(\choosenum)
						\task 5
						\task 20
						\task 40
					\end{tasks} \daan{B}{}

				\item 一般照明场所的线路允许电压损失为额定电压的( )。
					\begin{tasks}(\choosenum)
						\task $\pm$ 5\%
						\task $\pm$ 10\%
						\task $\pm$ 15\%
					\end{tasks} \daan{A}{}

				\item 热继电器的整定电流为电动机额定电流的( )\%。
					\begin{tasks}(\choosenum)
						\task 100
						\task 120
						\task 130
					\end{tasks} \daan{A}{}


				\item 铁壳开关在作控制电机启动和停止时,要求额定电流要大于或等于( )倍电动机的额定电流。
					\begin{tasks}(\choosenum)
						\task 一
						\task 两
						\task 三
					\end{tasks} \daan{B}{}



				\item 干粉灭火器可适用于( ) KV以下线路带点灭火。
					\begin{tasks}(\choosenum)
						\task 10
						\task 35
						\task 50
					\end{tasks} \daan{C}{}

					% \item N.A.
				% 	\begin{tasks}(\choosenum)
				% 		\task N.A.
				% 		\task N.A.
				% 		\task N.A.
				% 	\end{tasks} \daan{A}{}

					% \item N.A.
				% 	\begin{tasks}(\choosenum)
				% 		\task N.A.
				% 		\task N.A.
				% 		\task N.A.
				% 	\end{tasks} \daan{A}{}


					% \item N.A.
				% 	\begin{tasks}(\choosenum)
				% 		\task N.A.
				% 		\task N.A.
				% 		\task N.A.
				% 	\end{tasks} \daan{A}{}



			\end{enumerate}



