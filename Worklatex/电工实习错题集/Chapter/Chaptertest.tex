\documentclass[normal,cn,hang,blue,ctexfont,chinese]{mycls}                                                            

\RequirePackage{tcolorbox}
\RequirePackage{tikz} %% load tikz without tikz
\usetikzlibrary{backgrounds,calc,shadows,positioning,fit}
\newcommand*\circled[1]{\tikz[baseline=(char.base)]{
            \node[shape=circle,draw,inner sep=1pt] (char) {#1};}}

\newcounter{exam}[chapter]
\setcounter{exam}{0}
\renewcommand{\theexam}{\thechapter.\arabic{exam}}
\newenvironment{example}[1][]{
	 \refstepcounter{exam}
	 \par\noindent\textbf{\color{main}{\examplename} \theexam #1 }\rmfamily}{
	\par\ignorespacesafterend}

\addbibresource[location=local]{mybib.bib}

\begin{document}

\chapter{前言}
与传统的电励磁同步电机相比,永磁同步电机(Permanent Magnet Synchronous 
Motor,PMSM)具有结构简单、运行可靠、体积小、质量轻、损耗小、效率高,以及电机
的形状和尺寸可以灵活多样等显著优点 近年来,随着材料技术的不断发展,永磁材
料性能的不断提高,以及永磁电机控制技术的不断成熟, PMSM 已经在民用、航天和
军事等领域得到了广泛应用。然而, PMSM 是一个多变量、强糯合、非线性和变参数
的复杂对象,为了获得较好的控制性能,需要对其采用一定的控制算法 随着现代控
制理论的不断发展,近年来有关 PMSM 控制算法的研究已经成为研究热点,并已有
大量文献发表在国内外学术期刊和专著上。因此,为了使广大工程技术人员能够充
分了解、掌握和应用这一领域的最新技术,学会用 MATLAB 仿真软件进行相关
PMSM 控制算法的设计,作者编写了本书,以抛砖引玉,供广大读者学习参考。 

\begin{example}
	wo
\end{example}
本书是在总结作者多年研究成果的基础上,进一步理论化、系统化和实用化而形
成的;是基于目前较为先进的 MATLAB 2014b 仿真软件,在总体上按照由浅人深、
由易到难的原则进行编写的 。本书具有如下特点:

本书分为 部分共 10 章。第 部分为基础篇,包括第 章,第 章介绍
PMSM 的数学建模方法,第 章介绍 相电压源逆变器 PWM 技术,第 章介绍几
种常用的 PMSM 量控 MATLAB 仿真建模方法,第 章介绍 PMSM
直接转矩控 MATLAB 仿真建模方法。第 部分为进阶篇,包括第 章和
章,第 章介绍基于基波数学模型的 PMSM 无传感器控制 MATLAB 仿真
建模方法,第 章介绍基于高频信号注人的 PMSM 元传感器控制 MATLAB
真建模方法。第 部分为高级篇,包括第 10 章,第 章介绍六相 PMSM 的数学建
模方法,第 章介绍六相电压源逆变器 PWM 技术 MATLAB 仿真建模方法,第章介

\section{技术} 
	本章主要介绍 相电斥源逆变器 PWM 技术的基本原理和仿真建模,首先,介绍
	两电平空间矢量调制( Space Vector P ulse Width Modu ation,SVPWM )算法在线性
	调制区内的基本工作原理和实现方法,同时给出采用 Simulink 模块和 函数方法搭
	建的仿真模型,并给出仿真结果;其次,介绍几种常用的正弦脉宽调制( Sinusoidal 
	Pulse Width Modu ation,SPWM )算法的基本工作原理和 MATLAB 建模方法,并给
	出了仿真结果

		本书分为 部分共 10 章。第 部分为基础篇,包括第 章,第 章介绍
	PMSM 的数学建模方法,第 章介绍 相电压源逆变器 PWM 技术,第 章介绍几
	种常用的 PMSM 量控 MATLAB 仿真建模方法,第 章介绍 PMSM
	直接转矩控 MATLAB 仿真建模方法。第 部分为进阶篇,包括第 章和
	章,第 章介绍基于基波数学模型的 PMSM 无传感器控制 MATLAB 仿真
	建模方法,第 章介绍基于高频信号注人的 PMSM 元传感器控制 MATLAB
	真建模方法。第 部分为高级篇,包括第 10 章,第 章介绍六相 PMSM 的数学建
	模方法,第 章介绍六相电压源逆变器 PWM 技术 MATLAB 仿真建模方法,第章介

	\section{没有}  
	原因,我们2021年,正值“十四五”开局之年,也是中国共产党建党百年,总书记再到广西。围绕
	推动经济高质量发展、加快推进乡村振兴等问题,总书记深入调研,并指导开展党史学习教育。
	今年这次广西之行,是在全面贯彻党的二十大精神的开局之年再赴广西,也是中央经济工作会议
	后总书记首次国内考察。关键节点,再到广西,蕴含深意

		因此, 相对称正弦电压对应的空间电压矢量运动轨迹如图 - 所示。从图
	中可以看出,电压空间矢量 顶点的运动轨迹为一个圆,且以角速度 逆时针旋
	转。根据空间矢量变换的可逆性,可以想象若空间电压矢量 剧的顶点运动轨迹为 129 
	一个圆,则原 相电压越趋近于三相对称正弦波 。三相对称正弦电压供电是理想的
	供电方式,也是逆变器交流输出电压控制的追求目标。实际上,通过空间矢量变换,
	可以将逆变器 相输出的 个标量的控制问题转化为一个矢量的控制问题。
	\subsection{三项电机} 
	本章主要介绍 相电斥源逆变器 PWM 技术的基本原理和仿真建模,首先,介绍
	两电平空间矢量调制( Space Vector P ulse Width Modu ation,SVPWM )算法在线性
	调制区内的基本工作原理和实现方法,同时给出采用 Simulink 模块和 函数方法搭
	建的仿真模型,并给出仿真结果;其次,介绍几种常用的正弦脉宽调制( Sinusoidal 
	Pulse Width Modu ation,SPWM )算法的基本工作原理和 MATLAB 建模方法,并给
	出了仿真结果
		本书分为 部分共 10 章。第 部分为基础篇,包括第 章,第 章介绍
	PMSM 的数学建模方法,第 章介绍 相电压源逆变器 PWM 技术,第 章介绍几
	种常用的 PMSM 量控 MATLAB 仿真建模方法,第 章介绍 PMSM
	直接转矩控 MATLAB 仿真建模方法。第 部分为进阶篇,包括第 章和
	章,第 章介绍基于基波数学模型的 PMSM 无传感器控制 MATLAB 仿真
	建模方法,第 章介绍基于高频信号注人的 PMSM 元传感器控制 MATLAB
	真建模方法。第 部分为高级篇,包括第 10 章,第 章介绍六相 PMSM 的数学建
	模方法,第 章介绍六相电压源逆变器 PWM 技术 MATLAB 仿真建模方法,第章介


	
	

\chapter{学习联系}
章,第 章介绍基于基波数学模型的 PMSM 无传感器控制 MATLAB 仿真
	建模方法,第 章介绍基于高频信号注人的 PMSM 元传感器控制 MATLAB
	真建模方法。第 部分为高级篇,包括第 10 章,第 章介绍六相 PMSM 的数学建
	模方法,第 章介绍六相电压源逆变器 PWM 技术 MATLAB 仿真建模方法,第章介

	\begin{example}
		章,第 章介绍基于基波数学模型的 PMSM 无传感器控制 MATLAB 仿真
	建模方法,第 章介绍基于高频信号注人的 PMSM 元传感器控制 MATLAB
	真建模方法。第 部分为高级篇,包括第 10 章,第 章介绍六相 PMSM 的数学建
	模方法,第 章介绍六相电压源逆变器 PWM 技术 MATLAB 仿真建模方法,第章介
	\end{example}

	\begin{example}
		章,第 章介绍基于基波数学模型的 PMSM 无传感器控制 MATLAB 仿真
	建模方法,第 章介绍基于高频信号注人的 PMSM 元传感器控制 MATLAB
	真建模方法。第 部分为高级篇,包括第 10 章,第 章介绍六相 PMSM 的数学建
	模方法,第 章介绍六相电压源逆变器 PWM 技术 MATLAB 仿真建模方法,第章介
	\end{example}

	\begin{example}
		章,第 章介绍基于基波数学模型的 PMSM 无传感器控制 MATLAB 仿真
	建模方法,第 章介绍基于高频信号注人的 PMSM 元传感器控制 MATLAB
	真建模方法。第 部分为高级篇,包括第 10 章,第 章介绍六相 PMSM 的数学建
	模方法,第 章介绍六相电压源逆变器 PWM 技术 MATLAB 仿真建模方法,第章介
	\end{example}

	\mbox{frr}
	% % %%%%%%%%%%%%%%%%%%%%%%% 取消这段引用就有参考文献了%%%%%%%%%%%%%%%%%%%%%%%%%
		\newpage
	
	% % %%%%%%%%%%%%%%%%%%%%%%%%%%%%%%%%%%%%%%%%%%%%%%%%%%%%%%%%%%%%%%%%%%%%%%%%%
\end{document}