
\chapter{新能源车名词定义}
本章总结了新能源车中的名词定义:
\href{https://www.baidu.com}{百度一下} 

\section{专有名词}
    \begin{table}[!htbp]
        \renewcommand{\arraystretch}{1.3}
        \centering
        \caption{名称定义}
        \begin{tabular}{|p{6cm}|>{\centering\arraybackslash}p{2cm}|>{\centering\arraybackslash}p{3cm}|}  % >{\raggedleft\arraybackslash}p{2cm}居右
            \specialrule{0.2em}{0pt}{0pt} 
            \centering{单词}  & 缩写  & 中文\\    
            \hline
            Electric Vehicle  & EV  & 电动汽车\\
            Plug in Hybrid Electric Vehicles  & PHEV & 插电混动\\
            Society of Automotive Engineers   & SAE  & 国际汽车工程协会\\
            \hyperref[def1]{AC Level 1 Charging}  & - & 交流1级充电 \\
            \hyperref[def2]{AC Level 2 Charging}  & - & 交流2级充电 \\
            \hyperref[def3]{Charger}  & - & 充电器 \\
            \hyperref[def4]{Chassis Ground}  & - & 底盘接地 \\
            \hyperref[def5]{Conductive}  & - & 导电 \\
            \hyperref[def6]{Connector (Charge)}  & - & 连接器(充电) \\
            \hyperref[def7]{Contact (Charge)}  & - & 接触(充电) \\
            \hyperref[def8]{Control Pilot}  & CP & 控制导联 \\
            \hyperref[def9]{Coupler (Charge)}  & - & 耦合器(充电) \\
            \hyperref[def10]{DC Charging}  & - & 直流充电 \\
            \hyperref[def11]{Electric Vehicle}  & EV  & 电动汽车 \\
            \hyperref[def12]{Electric Vehicle Supply Equipment}  & EVSE  & 电动汽车供电设备 \\
            \hyperref[def13]{Equipment Ground (Grounding Conductor)}  & - & 设备接地(接地导体) \\
            \hyperref[def14]{EV/PHEV Charging System}  & - & EV/PHEV充电系统 \\
            \hyperref[def15]{Insulator}  & - & 绝缘体 \\
            \hyperref[def16]{Invalid Control Pilot}  & - & 无效控制导频 \\
            \hyperref[def17]{Off-Board Charger}  & - & 车外充电器 \\
            \hyperref[def18]{On-Board Charger}  & OBC & 车载充电器 \\
            \hyperref[def19]{Plug In Hybrid Electric Vehicle}  & PHEV & 插电混合动力汽车 \\
            \hyperref[def20]{Pre-Charge}  & - & 预充电 \\
            \hyperref[def21]{Vehicle Inlet (Charge)}  & - & 车辆进气口(充电) \\
            \specialrule{0.2em}{0pt}{0pt} 
        \end{tabular}
    \end{table}

\section{定义}



    \subsection*{AC LEVEL 1 CHARGING }
        \label{def1}
        A method that allows an EV/PHEV to be connected to the most common grounded electrical receptacles (NEMA 5-15R and 
        NEMA 5-20R). The vehicle shall be fitted with an on-board charger capable of accepting energy from the existing single 
        phase alternating current (AC) supply network. The maximum power supplied for AC Level 1 charging shall conform to the 
        values in Table 9. A cord and plug EVSE with a NEMA 5-15P plug may be used with a NEMA 5-20R receptacle. A cord and 
        plug EVSE with a NEMA 5-20P plug is not compatible with a NEMA 5-15R receptacle. 
            \begin{definition}{AC LEVEL 1 CHARGING}
                允许EV/PHEV连接到最常见的接地插座(NEMA 5-15R和NEMA 5-20R)的方法。车辆应配备车载充电器,能够接受来自现有单
                相交流电(AC)供电网络的能量。交流1级充电时提供的最大功率应符合表9中的数值。带NEMA 5-15P插头的电缆和插头EVSE可
                与NEMA 5-20R插座一起使用。带NEMA 5-20P插头的电源线和插头EVSE与NEMA 5-15R插座不兼容。
            \end{definition}

        
    \subsection*{AC LEVEL 2 CHARGING }
        \label{def2}
        A method that uses dedicated AC EV/PHEV supply equipment in either private or public locations. The vehicle shall be 
        fitted with an on-board charger capable of accepting energy from single phase alternating current (AC) electric vehicle 
        supply equipment. The maximum power supplied for AC Level 2 charging shall conform to the values in Table 9. 
            \begin{definition}{AC LEVEL 2 CHARGING}
                一种在私人或公共场所使用专用交流EV/PHEV供电设备的方法。车辆应配备车载充电器,能够接受单相交流电(AC)电动车辆供电设备的能量。交流2级充电时提供的最大功率应符合表9中的数值
            \end{definition}
    
    \subsection*{CHARGER }
        \label{def3}
        An electrical device that converts alternating current energy to regulated direct current for replenishing the energy of a 
        rechargeable energy storage device (i.e., battery) and may also provide energy for operating other vehicle electrical 
        systems.  
            \begin{definition}{CHARGER}
                一种将交流电能量转换为可调节直流电的电气装置,用于补充可充电储能装置(即电池)的能量,并且还可以为操作其他车辆电气系统提供能量。
            \end{definition}

   \subsection*{CHASSIS GROUND }
        \label{def4}
        The conductor used to connect the non-current carrying metal parts of the vehicle high voltage system to the equipment 
        ground. 
            \begin{definition}{CHASSIS GROUND}
                用于将车辆高压系统的非载流金属部件与设备接地相连接的导体。
            \end{definition}

    \subsection*{CONDUCTIVE }
        \label{def5}
        Having the ability to transmit electricity through a physical path (conductor). 
            \begin{definition}{CONDUCTIVE}
                具有通过物理路径(导体)传输电的能力的。
            \end{definition}

    \subsection*{CONNECTOR (CHARGE) }
        \label{def6}
        A conductive device that by insertion into a vehicle inlet establishes an electrical connection to the electric vehicle for the 
        purpose of transferring energy and exchanging information. This is part of the coupler. 
        \begin{definition}{CONNECTOR (CHARGE)}
            一种导电装置,其通过插入车辆进气道与电动车辆建立电气连接,以传输能量和交换信息。这是耦合器的一部分。
        \end{definition}

    \subsection*{CONTACT (CHARGE) }
        \label{def7}    
        A conductive element in a connector that mates with a corresponding element in the vehicle inlet to provide an electrical 
        path. 
        \begin{definition}{CONTACT (CHARGE)}
            连接器中的导电元件,其与车辆入口中的相应元件配合以提供电路径。
        \end{definition}


        \subsection*{CONTROL PILOT}
        \label{def8}   
            An electrical signal that is sourced by the Electric Vehicle Supply Equipment (EVSE). Control Pilot is the primary control conductor and is connected to the equipment ground through control circuitry on the vehicle and performs the following functions: 
        \begin{enumerate}
            \item Verifies that the vehicle is present and connected 
            \item Permits energization/de-energization of the supply 
            \item Monitors the presence of the equipment ground
            \item Establishes vehicle ventilation requirements
        \end{enumerate}
        \begin{definition}{CONTROL PILOT}                        
        由电动车辆供电设备(EVSE)提供的电信号。控制导联(Control Pilot)是主要的控制导体,通过车辆上的控制电路与设备接地连接,实现以下功能:
        a.验证车辆是否存在并连接
        b.允许电源上电/下电
        c.将供电设备额定电流传输给车辆
        d.监控设备接地是否存在
        e.建立车辆通风要求
        \end{definition}
        
        \subsection*{COUPLER (CHARGE)}
        \label{def9}
        A mating vehicle inlet and connector set.
        \begin{definition}{COUPLER (CHARGE)}
        配套的车辆进气口和连接器组。
        \end{definition}
        
        \subsection*{DC CHARGING}
        \label{def10}
        A method that uses dedicated direct current (DC) EV/PHEV supply equipment to provide energy from an appropriate offboard charger to the EV/PHEV in either private or public locations.
        \begin{definition}{DC CHARGING}
        一种使用专用直流(DC) EV/PHEV供电设备,从合适的车载充电器向私人或公共场所的EV/PHEV提供能量的方法。
        \end{definition}
        
        \subsection*{ELECTRIC VEHICLE (EV)}
        \label{def11}
        An automotive type vehicle, intended for highway use, primarily powered by an electric motor that draws from a rechargeable energy storage device. For the purpose of this document the definition in the United States Code of Federal Regulations – Title 40, Part 600, Subchapter Q is used. Specifically, an automobile means:
        \begin{enumerate}[label=\alph*.]
            \item Any four-wheeled vehicle propelled by a combustion engine using on-board fuel or by an electric motor drawing current from a rechargeable storage battery or other portable energy devices (rechargeable using energy from a source off the vehicle such as residential electric service).
            \item Which is manufactured primarily for use on public streets, roads, and highways.
            \item Which is rated not more than 3855.6 kg (8500 lb), which has a curb weight of not more than 2721.6 kg (6000 lb), and which has a basic frontal area of not more than 4.18 m\(^2\) (45 ft\(^2\)).
        \end{enumerate}
        \begin{definition}{ELECTRIC VEHICLE (EV)}
        一种用于高速公路的汽车型车辆,主要由从可充电能量存储装置中提取的电动机提供动力。本文档使用美国联邦法规第40卷第600部分Q分章中的定义。具体来说,汽车是指:\\
        a.任何由使用车载燃料的内燃机或由从可充电蓄电池或其他便携式能源设备(可使用来自车辆外的能源,如住宅电力服务)获取电流的电动机推动的四轮车辆。\\
        b.主要用于公共街道、道路和高速公路的。\\
        c.额定重量不超过3855.6公斤(8500磅),整备重量不超过2721.6公斤(6000磅),基本正面面积不超过4.18平方米(45平方英尺)。\\
        \end{definition}
        
        \subsection*{ELECTRIC VEHICLE SUPPLY EQUIPMENT (EVSE)}
        \label{def12}
        The conductors, including the ungrounded, grounded, and equipment grounding conductors, the electric vehicle connectors, attachment plugs, and all other fittings, devices, power outlets, or apparatuses installed specifically for the purpose of delivering energy from the premises wiring to the electric vehicle. Charging cords with NEMA 5-15P and NEMA 5-20P attachment plugs are considered EVSEs.
        \begin{definition}{ELECTRIC VEHICLE SUPPLY EQUIPMENT (EVSE)}
        导体,包括不接地、接地和设备接地导体,电动汽车连接器、附件插头以及所有其他配件、设备、电源插座或专门为从房屋布线向电动汽车输送能量而安装的器具。带有NEMA 5-15P和NEMA 5-20P连接插头的充电线被认为是EVSE。
        \end{definition}
        
        \subsection*{EQUIPMENT GROUND (GROUNDING CONDUCTOR)}
        \label{def13}
        A conductor used to connect the non-current carrying metal parts of the EV/PHEV supply equipment to the system grounding conductor, the grounding electrode conductor, or both, at the service equipment.
        \begin{definition}{EQUIPMENT GROUND (GROUNDING CONDUCTOR)}
        一种导体,用于将EV/PHEV供电设备的非载流金属部件连接到服务设备的系统接地导体、接地电极导体或两者之间。
        \end{definition}
        
        \subsection*{EV/PHEV CHARGING SYSTEM}
        \label{def14}
        The equipment required to condition and transfer energy from the constant frequency, constant voltage supply network to the direct current, variable voltage EV/PHEV traction battery bus for the purpose of charging the battery and/or operating vehicle electrical systems while connected.
        \begin{definition}{EV/PHEV CHARGING SYSTEM}
        从恒频、恒压供电网络调节和传输能量到直流、变电压EV/PHEV牵引电池母线所需的设备,以便在连接时为电池充电和/或操作车辆电气系统。
        \end{definition}
        
        \subsection*{INSULATOR}
        \label{def15}
        The portion of a charging system that provides for the separation, support, sealing, and protection from live parts.
        \begin{definition}{INSULATOR}
        充电系统的一部分,提供与带电部件的分离、支撑、密封和保护。
        \end{definition}
        
        \subsection*{INVALID CONTROL PILOT}
        \label{def16}
        A Control Pilot outside of the frequency definition of Table 3 or any Control Pilot duty cycle which is defined as an error state in Table 5.
        \begin{definition}{INVALID CONTROL PILOT}
        在表3的频率定义之外的控制导频或表5中定义为错误状态的任何控制导频占空比。
        \end{definition}
        
        \subsection*{OFF-BOARD CHARGER}
        \label{def17}
        A charger located off of the vehicle.
        \begin{definition}{OFF-BOARD CHARGER}
        位于车辆外的充电器。
        \end{definition}
        
        \subsection*{ON-BOARD CHARGER}
        \label{def18}
        A charger located on the vehicle.
        \begin{definition}{ON-BOARD CHARGER}
        位于车辆上的充电器。
        \end{definition}
        
        \subsection*{PLUG IN HYBRID ELECTRIC VEHICLE (PHEV)}
        \label{def19}
        A hybrid vehicle with the ability to store and use off-board electrical energy in a rechargeable energy storage device.
        \begin{definition}{PLUG IN HYBRID ELECTRIC VEHICLE (PHEV)}
        一种能够在可充电储能装置中存储和使用车载电能的混合动力汽车。
        \end{definition}
        
        \subsection*{PRE-CHARGE}
        \label{def20}
        Pre-charge circuits are designed to limit the electrical inrush current into the bulk capacitors prior to enabling the entire high voltage system. High inrush current can stress and damage the capacitors and other components on the high voltage DC bus such as fuses, input filters and power modules. Pre-charge circuits are typically comprised of a resistor and high voltage contactor.
        \begin{definition}{PRE-CHARGE}
        预充电电路的设计是为了在启动整个高压系统之前限制进入大块电容器的电涌电流。高浪涌电流会对高压直流母线上的电容器和其他部件造成压力和损坏,如熔断器、输入滤波器和电源模块。预充电电路通常由电阻和高压接触器组成。
        \end{definition}
        
        \subsection*{VEHICLE INLET (CHARGE)}
        \label{def21}
        The device on the electric vehicle into which the connector is inserted for the purpose of transferring energy and exchanging information. This is part of the coupler.
        \begin{definition}{VEHICLE INLET (CHARGE)}
        用于传递能量和交换信息的将连接器插入其中的电动汽车上的装置。这是耦合器的一部分。
        \end{definition}




