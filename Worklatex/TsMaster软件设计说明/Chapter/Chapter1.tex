\chapter{TsMaster软件设计}

本文基于TsMaster软件设计上位机程序。


\section{TsMaster软件}
TSMaster是一款虚拟仪器软件平台,可连接、配置并控制所有同星的硬件工具、设备,实
现汽车总线嵌入式代码生成、监控、仿真、开发、诊断、标定、ECU刷写、I/O控制、测试测量等功能。\\
\textbf{\textcolor{red}{主要应用场景:}}
\begin{itemize}
	\kaishu
	\item 汽车总线通讯系统设计。
	\item 整车网络验证。
	\item 零部件PV/DV验证。
	\item 路试标定。
	\item 信息安全验证,远程DOS攻击。
	\item ADAS无人驾驶数据采集回放。
\end{itemize}
相较于CANoe软件,该软件不需要测试


\section{报文预发送}
\begin{lstlisting}[language=c]
	#include "studio"

	int main()
	{
		return 0;
	}
	char DSPraw[8];
	char MCUraw[4];
	   
   // 
	if (ACAN->FData[0] == 0x10 && ACAN->FData[1] == 0x13 &&  ACAN->FData[2] == 0x62
	&& ACAN->FData[3] == 0xF1 &&  ACAN->FData[4] == 0x94)
	{
		   MCUraw[0]= (ACAN->FData[5]&0x0F)+0x30;
		   MCUraw[1]= (ACAN->FData[6]&0x0F)+0x30;
		   MCUraw[2]= (ACAN->FData[7]&0x0F)+0x30;
		   MCUraw[3]= '\0'; 
		   MCUVersion.set(MCUraw);
		  }
		  
   if (ACAN->FData[0] == 0x21)
   {
		   DSPraw[0]= (ACAN->FData[1] & 0x0F) + 0x30;
		   DSPraw[1]= (ACAN->FData[2] & 0x0F) + 0x30;
		   DSPraw[2]= (ACAN->FData[3] & 0x0F) + 0x30;
		   DSPraw[3]= (ACAN->FData[4] & 0x0F) + 0x30;
		   DSPraw[4]= (ACAN->FData[5] & 0x0F) + 0x30;
		   DSPraw[5]= (ACAN->FData[6] & 0x0F) + 0x30;
		   DSPraw[6]= (ACAN->FData[7] & 0x0F) + 0x30;
		   DSPraw[7]= '\0';
		   DSPVersion.set(DSPraw); 
   
	}
   
\end{lstlisting}

\begin{lstlisting}[language=matlab]
	clear all
	clc
	if a==5;
		ret = DspVersion.set(360);
	end
\end{lstlisting}










