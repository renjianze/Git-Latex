\chapter{判断题(易错题)}





    \begin{enumerate}
        \kaishu
        \item \Fquestion{ 单相220V电源供电的电气设备,应选用三极式漏电保护装置。}\\ \jiexi{二极两线或单极两线。}
        \item \Fquestion{ 从过载角度出发,规定了熔断器的额定电压。}
        \item \Fquestion{ 锡焊晶体管等弱电元件应用100W的电烙铁。}
        \item \Tquestion{ 额定电压为380V的熔断器可用在220V的线路中。}
        \item \Fquestion{ 低压验电器可以验出500V以下的电压。}\\ \jiexi{100V以下就验不出了。}
        \item \Fquestion{ 对电机轴承润滑的检查,可通电转动电动机转轴,看是否转动灵活,听有无异声。}
        \item \Fquestion{ RCD的额定动作电流是指能时RCD动作的最大电流。}
        \item \Fquestion{ RCD的中性线可以接地}
        \item \Tquestion{ RCD的选择,必须考虑用电设备和电路正常泄露电流的影响。}
        \item \Fquestion{ 绝缘老化是一种化学老化。}  \\ \jiexi{物理和化学老化}
        \item \Fquestion{ SELV只作为接地系统的电击保护。}
        \item \Fquestion{ 使用改变磁极对数来调速的电机一般都是绕线型转子电动机。}\\ \jiexi{鼠笼式异步电机}
        \item \Fquestion{ 转子串频敏变阻器启动的转矩大,{\color{red}适合重载启动}。}  \\ \jiexi{适用于轻载}
        \item \Tquestion{ 改变转子电阻调速这种方法只适用于绕线式异步电动机。} 
        \item \Fquestion{ 如果电容器运行时,检查发现温度过高,应加强通风。}
        \item \Fquestion{ 目前我国生产的接触器额定电流一般大于或等于630A。}  \\ \jiexi{小于或等于630A。}
        \item \Tquestion{ 使用万用表测量电阻,每换一次欧姆档都要进行欧姆调零。}
        \item \Tquestion{ 跨越铁路、公路等的架空绝缘铜导线截面不小于16 $\rm mm^2$}
        \item \Fquestion{ 雷击产生的高电压和耀眼的白光可对电气装置和建筑物及其他设施造成毁坏,电力设施或电力线路遭破环可能导致大规模停电。} \\ \jiexi{耀眼光芒不破环设备}
        \item \Fquestion{ 在设备运行中,发生起火的原因电流热量是间接原因,而火花或电弧则是直接原因。} \\ \jiexi{原因是过载或短路}
        \item \Fquestion{ 中间继电器实际上是一种动作与释放值可调节的电压继电器。}  \\ \jiexi{中间继电器的动作与释放值在制造过程中已经被确定,是不能调节的}
        \item \Fquestion{ 雷电后造成架空线路产生高压冲击波,这种雷电称为直击雷。}  \\ \jiexi{“感应电压”}
        \item \Tquestion{ 当电气火灾发生时首先应迅速切断电源,在无法切断电源的情况下,应迅速选择{\color{red}干粉、$\rm CO_2$等不导电的灭火器材}进行灭火。}
        \item \Tquestion{ 除独立避雷针之外,在接地电阻满足要求的前提下,防雷接地装置可以和其他接地装置共用。}
        \item \Fquestion{ 接了漏电开关之后,设备外壳就不需要在接地或接零了。}  \\ \jiexi{必须要接地}
        \item \Fquestion{ 热继电器的双金属片弯曲的速度与{\color{red}电流大小}有关,电流越大,速度越快,这种特性称正比时限特性。}  \\ \jiexi{与温度有关。}
        \item \Fquestion{ 接地电阻测试仪就是测量线路的{\color{red}绝缘电阻}的仪器。}  \\ \jiexi{接地电阻}
        \item \Fquestion{ 热继电器是利用{\color{red}双金属片受热弯曲}而推动触点动作的一种保护电器,它主要用于线路的速断保护。}  \\ \jiexi{电流通过发热元件时产生的热。}
        \item \Fquestion{ 万能转换开关的定位结构一般采用滚轮式卡转轴辐射型结构} \\ \jiexi{暂无解析} 
        \item \Fquestion{ 脱离电源后,触电者神志清醒,应让触点者来回走动,加强血液循环。} \\ \jiexi{安静休息1-2 h,并严密观察}
        \item \Tquestion{ 触电者神志不清,有心跳,但呼吸停止,应立即进行口对口人工呼吸。}
        \item \Fquestion{ 为安全起见,更换熔断器时,最好断开负载。}
        \item \Fquestion{ 自动开关属于手动电器。} \\ \jiexi{自动开关不属于手动电器。}
        \item \Fquestion{ 在采用多级熔断器保护中,{\color{red}后级熔体的额定电流比前级大},以电源端为最前端。}  \\ \jiexi{比前级小。}
        \item \Fquestion{ 交流接触器常见的额定最高工作电压达到6000 V。} \\ \jiexi{工作电压380 V,矿用的660 V}
        \item \Fquestion{ 剩余动作电流小于或等于0.3 A的RCD属于高灵敏度RCD。} \\ \jiexi{能使RCD动作的最小电流。}
        \item \Fquestion{ 在三相交流电路中,负载为星形接法时,其相电压等于三相电源的线电压。} \\ \jiexi{Y: 线电压 = $\sqrt{3}$相电压 \hspace{1.5em} 线电流 = 相电流 \\ 0 \quad \quad  \quad$\triangle$: 线电压 = 相电压  \hspace{1.5em} 线电流 = $\sqrt{3}$相电流。 }
        \item \Fquestion{ 移动电气设备电源应采用高强度铜芯{\color{red}橡皮护套硬绝缘电缆}。} \\ \jiexi{软护套电缆。}
        \item \Fquestion{ 为了安全, 高压线路通常采用绝缘导线。} \\ \jiexi{10 kV的城区高压配电}
        \item \Fquestion{ 为保证零线安全,三相四线的零线必须加装熔断器。} \\ \jiexi{零线禁止加熔断器。} 
        \item \Tquestion{ III类电动工具的工作电压不超过50 V。}
        \item \Fquestion{ II类和III类设备都要采取接地或接零措施。}  \\ \jiexi{II 类不需要接地。}
        \item \Fquestion{ 对于异步电动机,国家标准规定3 kW以下的电动机均采用三角形联接。} \\ \jiexi{3 kW以上需要三角形联接。}
        \item \Fquestion{ 概率为50\%时,成年男性的平均感知电流值约为1.1 mA,最小为 0.5 mA,成年女性约为0.6 mA。}
        \item \Fquestion{ 并联补偿电容器主要用在直流电路中。}  \\ \jiexi{交流电路中。}
        \item \Fquestion{ 导线接头位置应尽量在绝缘子固定处,以方便统一扎线。}  \\ \jiexi{固定处0.5 m以上}
        \item \Tquestion{ 再生发电制动只用于电动机转速高于同步转速的场合}
        \item \Fquestion{ 一般情况下,接地电网的单相触电比不接地的电网的危险性小} \\ \jiexi{危害大}
        \item \Fquestion{ 导线接头的抗拉强度必须与原导线的抗拉强度相同。} \\ \jiexi{接头处强度不低于原导线的90 \%,电阻不超过原导线1.2倍。}
        \item \Tquestion{ {\color{red}\ding{72}}钳形电流表既可以测交流电流,也能测量直流电流。}
        \item \Fquestion{ {\color{red}\ding{72}}交流钳形电流表可测量交直流电流。}
        \item \Fquestion{ {\color{red}\ding{72}}直流电流表可以用于交流电路测量。}
        \item \Tquestion{ \ding{111}保护接零适用于中性点直接接地的配电系统中。}
        \item \Fquestion{ 断路器在选用时,要求断路器的额定通断能力要大于或等于被保护线路中可能出现的最大负载电流。} \\ \jiexi{断路器的额定分断能力大于线路中预期最大短路电流。通断能力一般指额定电流。}
        \item \Fquestion{ 隔离开关是指承担接通和断开{\color{red}电流}任务,将电路与电源隔开。} \\ \jiexi{断开电源任务。}
        \item \Fquestion{ 10 kV以下运行的阀型避雷器的绝缘电阻每年测量一次。}\\ \jiexi{半年或三个月。}
        \item \Fquestion{ 30 Hz $\backsim$ 40 Hz的电流危险性最大。 }
        \item \Fquestion{ 对称的三相电源是由振幅相同、初相依次相差$120^{\circ}$的正弦电源,连接组成的供电系统。 {\color{white}00}} \\ \jiexi{同时产生振幅相同。}
        \item \Fquestion{ 在高压线路发生火灾时,应采用有相应绝缘等级的绝缘工具,迅速拉开隔离开关切断电源,选择二氧化碳或干粉灭火器进行灭火。} \\ \jiexi{先断开断路器,再拉开隔离开关,再灭火。}
        \item \Fquestion{ 电工作业分为高压电工和低压电工。} \\ \jiexi{高压、低压和防爆电气。}
        \item \Fquestion{ 在断电之后,电动机停转,当电网再次来电,电动机能自行起动的运行方式称为失压保护。}
        \item \Tquestion{ 组合开关在选作直接控制电机时,要求其额定电流可取电动机额定电流的2 $\backsim$ 3倍。}
        \item \Fquestion{ 两相触电危险性比单相触电小。}
        \item \Fquestion{ {\color{red}\ding{115}}危险场所室内的吊灯与地面距离不少于3 m。}
        \item \Fquestion{ {\color{red}\ding{115}}当灯具达不到最小高度时,应采用24 V以下电压。} \\ \jiexi{必须达到某一高度。}
        \item \Fquestion{ {\color{red}\ding{115}}吊灯安装在桌子上方时,与桌子的垂直距离不少于1.5 m。} \\ \jiexi{0.5 m至2.2 m。}
        \item \Fquestion{ 手持电动工具有两种分类方式,即按工作电压分类和按防潮程度分类。} 
        \item \Fquestion{ 根据用电性质,电力线路可分为动力线路和配电线路。} \\ \jiexi{有不同的方式分。}
        \item \Tquestion{ TT系统是配电网中性点直接接地,用电设备外壳也采用接地措施的系统。}
        \item \Tquestion{ 在直流电路中,常用棕色表示正极} \\ \jiexi{直流:棕色是正极,蓝色是负极;交流:棕色是火线,蓝色是零线。}
        \item \Fquestion{ 电度表是专门用来测量设备功率的装置} \\ \jiexi{设备使用功率总和的装置。}
        \item \Fquestion{ 高压水银灯的电压比较高,所以称为高压水银灯。}
        \item \Fquestion{ 通电时间增加,人体电阻因出汗而增加,导致通过人体的电流减小。}
        \item \Tquestion{{\color{red}\ding{77}\ding{78}\ding{79}\ding{80}\ding{81}} 螺口灯台的台灯应采用三孔插座。}
        \item \Fquestion{{\color{red}\ding{72}\ding{73}\ding{74}\ding{75}\ding{76}}为了安全起见,更换熔断器时,最好断开负载。}
    \end{enumerate}








    

% \chapter{看图找错}

